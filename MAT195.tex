\documentclass[a4paper,12pt]{report}

\usepackage{amsmath,amsfonts,mathtools}

\begin{document}
\title{MAT195 TextBook Notes}
\author{Aman Bhargava}
\date{January 2019}
\maketitle

\tableofcontents


\pagebreak
\section{Review: Memorizy Stuff}
\subsection{Trig Function Derivatives}
\def\arraystretch{2}%
\begin{tabular}{cc}
$ \frac{d}{dx}sin(x) = cos(x) $ & $ \frac{d}{dx}csc(x) = -csc(x)cot(x) $ \\
$ \frac{d}{dx}cos(x) = -sin(x) $ & $ \frac{d}{dx}sec(x) = sec(x)tan(x) $ \\
$ \frac{d}{dx}tan(x) = sec^2(x) $ & $ \frac{d}{dx}cot(x) = -csc^2(x) $ \\
\end{tabular}

\subsection{Inverse Trig Derivatives}
\def\arraystretch{2}%
\begin{tabular}{cc}
$ \frac{d}{dx}sin^{-1}(x) = \frac{1}{\sqrt(1-x^2)} $ \\
$ \frac{d}{dx}cos^{-1}(x) = \frac{-1}{\sqrt(1-x^2)} $ \\
$ \frac{d}{dx}tan^{-1}(x) = \frac{1}{1+x^2} $ \\
\end{tabular}

\subsection{How to complete the Square}
\begin{enumerate}
\item Put $ax^2 + bx$ in brackets and forcefully factor out the $a$
\item Add $ ( \frac{b}{2})^2$ to the inside of the brackets and subtract it from the outside (you got it)
\item Factor and be happy that you've completed the square;
\end{enumerate}

\subsection{Trig Angle Sums}
\begin{enumerate}
\item $sin(A+B) = sin(A)cos(B) + cos(A)sin(B)$
\item $cos(A+B) = cos(A)cos(B) - sin(A)sin(B)$
\item $sin(A-B) = sin(A)cos(B) - cos(A)sin(B)$
\item $cos(A-B) = cos(A)cos(B) + sin(A)sin(B)$
\end{enumerate}

\subsection{Hyperbolic Trig Functions}
\subsection{Inverse Hyperbolic Trig Function}

\section{Introduction and Course Description}
\chapter{Techniques of Integration (Chapter 7 in Textbook)}
\section{Integration by Parts}

Integration by parts is basically just the reverse product rule.\\

Product rule: $d/dx [f(x)g(x)] = f'(x)g(x) + f(x)g'(x)$\\

You could reverse this simply, but it wouldn't be that useful. The more useful form that \textit{is} the integration by parts formula looks like this: $$\int [f(x)g'(x)] = f(x)g(x) - \int [f'(x)g(x)]$$

You can think through that one pretty easily | you are just splitting up the initial integral, then moving half of it to the other side.\\

That's pretty useful, but there's an even more useful way to write the formula, and it looks like this: $$\int u dv = uv - \int v du$$

This works because we let $u = f(x)$ and $v = g(x)$. $g'(x) = v' = dv/dx$, and same with $u'$. So to get to that formula we go like this:
\begin{eqnarray}
\int uv' dx = uv - \int u'v dx \\
\int u \frac{dv}{dx} dx = uv - \int v \frac{du}{dx} dx \\
\int u dv = uv - \int v du
\end{eqnarray}

\subsection{Tips for Integration by Parts:}
\begin{itemize}
\item When using the $uv$ equation, it's useful to define things in this order:
\begin{itemize}
\item u = ? dv = ?
\item du = ? v = ?
\item keeping in mind that $u'dx = du$ and $\int \frac{dv}{dx} dx = \int v' = v$
\end{itemize}
\item Choose your $u$ so that it becomes simpler when differentiated, and let your $v$ be the thing that gets a little hairier.
\item Practice a lot from the textbook, ya dingus
\end{itemize}

\section{Trigonometric Integrals}
There are a bunch of configurations of trig functions for which we need to learn the steps necessary to take the integral. It's important to practice this because recognizing the form of the integral is the most difficult thing to do here.
\subsection{Strategy for $\int sin^m (x)cos^n (x)$}
\paragraph{If the power of cosine is odd:}
"Save" one of the cosine terms, and then express it as $\int sin^m(x)cos^{2k+1}(x)*cos(x)$. Then turn the $cos^{2k+1}(x)$ into sin terms with pythagorean identity. Then substitute $u = sin(x)$ and solve.
\paragraph{If the power of sine is odd:} 
Do the same thing but reverse $sin$ and $cos$ (save one $sin(x)$ and sub $u$ for $cos(x)$
\paragraph{If both powers are even:} 
Use the following identities to help you solve it:
\begin{eqnarray}
sin^2 (x) = 1/2 (1-cos(2x)) \\
cos^2 (x) = 1/2 (1+cos(2x)) \\
sinxcosx = 1/2 sin2x
\end{eqnarray}

\subsection{Strategy for $\int tan^m (x)sec^n (x)$}
\paragraph{If the power of $sec(x)$ is even,} "save" a factor of $sec^2 (x)$ and use identity $sec^2 (x) = 1+tan^2 (x)$ to express the rest in terms of $tan(x)$. Then substitute $u = tan(x)$
\paragraph{If the power of tangent is odd,} save a factor of $sec(x)tan(x)$ and convert the rest of the $tan(x)$'s using $tan^2(x) = sec^2 (x) - 1$ 
\paragraph{Also note the following:} $$\int tan(x) dx = ln|sec(x)| + C$$ $$\int sec(x) dx = ln|sec(x) + tan(x)| + C$$
\paragraph{Remember this as well:} $$\frac{d}{dx} tan(x) = sec^2(x)$$ $$\frac{d}{dx} sec(x) = sec(x)tan(x)$$

\subsection{Strategy for $\int sin(mx)cos(nx) dx$}
Use the following identities:
\begin{eqnarray}
sin A cos B = 1/2 [sin(A-B) + sin(A+B)] \\
sin A sin B = 1/2 [cos(A-B) - cos(A+B)] \\
cos A cos B = 1/2 [cos(A-B) + cos(A+B)]
\end{eqnarray}

\subsection{Strategy for $\int csc^m (x)cot^n (x) dx$}
Know the following things:
\begin{eqnarray}
\frac{d}{dx} csc(x) = -csc(x)cot(x) \\
cot^2(x) = csc^2(x) - 1 \\
\frac{d}{dx} cot(x) = -csc^2(x)
\end{eqnarray}

\section{Trig Sub}
\paragraph{What is trig sub?} Trig sub is when you use the \textit{inverse substitution} rule in conjunction with useful trigonometric identities and trigonometric integrals to solve integrals that you wouldn't otherwise be able to solve.
\subsection{Inverse Substitution}
Unlike u-substitution, you are substituting in a non-equivalent function ($g(x)$) for x instead of substituting a variable like $u$ for the actual value of x. Hence, the following result arises: $$\int f(x) dx = \int f(g(x))g'(x) dx $$
\paragraph{Qualificiations:} $g$ must have an inverse function, and $g$ must be one-to-one.
\subsection{List of Trig Subs}
The main use of trig subs is to get rid of irritating radical signs that make integration hard. The following is a table of types (from the textbook):

\medskip
\begin{tabular}{c|l|c}
Expression & Substitution & Identity \\
\hline
$\sqrt{a^2 - x^2}$ & $x = a sin \theta$, $-\pi/2 <= \theta <= \pi/2$ & $1-sin^2 \theta = cos^2 \theta$ \\
$\sqrt{a^2 + x^2}$ & $x = a tan \theta$, $-\pi/2 < \theta < \pi/2$ & $1+tan^2 \theta = sec^2 \theta$ \\
$\sqrt{x^2 - a^2}$ & $x = a sec \theta$, $-\pi/2 <= \theta <= \pi/2$ & $sec^2 \theta - 1= tan^2 \theta$ \\
\end{tabular}

\subsection{General Layout for Trig Sub}
\begin{enumerate}
\item Make sure there is no other way (e.g. u-sub, etc.)
\item If there is a quadratic in the root, complete the square
\item Recognize the stuff in the root as one of the three.
\item Set $x = a*trig(\theta)$ and find $dx$ in terms of $\theta$
\item Solve the stuff.

\end{enumerate}

\section{Partial Fractions}
This is just a way to break up rational functions in to little pieces that we can actually deal with.
A proper rational function is one where the power on the top polynomial is lower than that of the bottom. Improper rationals are the other way around.

In order to use partial fractions, you need to make the rational function a proper one.

There are a few cases to consider for splitting things up into partial fractions:

\subsection{Denominator is only distinct Linear Factors}
\begin{enumerate}
\item Set an equality between the original rational and $\frac{A}{(root_1)} + \frac{B}{(root2)} + ...$
\item Multiply both sides by the denominator of the rational
\item Expand and solve for $A, B, ...$
\end{enumerate}

\subsection{Denominator is only Linear Factors but some are Repeated}
It's roughly the same as last time, EXCEPT:
\begin{enumerate}
\item Suppose $(root_1)$ is repeated $k$ times so $(root_1)^k$ is a factor
\item Then you need to use $\frac{A_1}{(root_1)} + \frac{A_2}{(root_1)^2} + ... + \frac{A_k}{(root_1)^k}$
\item Now solve as you did last time.
\end{enumerate}

\subsection{Denominator has non-repeated Quadratic Factors}
Basically just have a linear term on top in the expansion, so $\frac{A_1x+B}{ax^2+bx+c}$ would be a term in the thing.

\paragraph{Also, } $\int\frac{dx}{x^2+a^2} = \frac{1}{a}tan^{-1}(\frac{x}{a})+C$

\subsection{Denominator has repeated Quadratic Factors}
Roughly the same as when you have linear repeated factors. if you have $(root_1)^k$, then you get $$\frac{Ax+B}{(root_1)} + \frac{Cx+D}{(root_1)^2} + ... + \frac{Yx+Z}{(root_1)^k}$$
Then just solve as before.

\subsection{Rationalizing Substitutions}
Use substitutions to make annoying functions into rational functions and solve from there. For instance, if you have a radical in the numerator, make the subsitution $u^2 = $ whatever was in the radical.

\section{Strategy for Solving Integrals}
\begin{enumerate}
\item Simplify the integrand with identities/algebra if possible.
\item Look for an obvious u-sub
\item Classify the integrand according to its form:
\begin{enumerate}
\item Trig integral ($sin^m(x)cos^n(x)$, etc.)
\item Rational Function
\item Integration by parts - especially with $polynomial * transcendental function$
\item Radicals $\to$ consider trig sub ($x = atan\theta$, etc.)
\end{enumerate}
\item Try again, using several methods, and drawing from your MASSIVE PAST EXPERIENCE

\end{enumerate}


\section{Improper Integrals (Involving Infinity)}
\subsection{Definition of Improper Integral}
$\int_{a}^{\infty} f(x) dx = \lim_{t \to \infty} \int_a^t f(x) dx $
Convergent improper integrals have an actual value. Divergent improper integrals don't.
Integrals from negative infinity to positive infinity exist if and only if the integral from -infinity to $a$ converges and the integral from $a$ to +infinity converges (it is the sum of the two).

\paragraph{For $\int_1^{\infty} \frac{1}{x^p} dx$} is convergent if $p > 1$

\paragraph{For integrals where the y value goes to infinity}, then you take limits around where it goes to infinity, ya dingus.

\subsection{Comparison Theorem}
You can prove that an integral is convergent if you can find another one for which the function is strictly of greater magnitude that is convergent (and likewise for non-convergency)

\chapter{Further Applications of Integration}
\section{Arc Length}
\subsection{Definition of Arc Length}
Arc length is the actual length of a curve. It is defined as: $$L = \lim_{n \to \infty} \sum_{i = 1}^{n} |P_{i-1},P_i|$$

\subsection{Arc Length Formula}
$$L = \int_a^b \sqrt{1 + [f'(x)]^2} dx $$

\section{Surface Area of Revolution}
\subsection{Surface Area of Revolution Formulae}
$$S = \int_a^b 2\pi y \sqrt{1+(\frac{dy}{dx})^2}dx$$
$$S = \int_a^b 2\pi y \sqrt{1+(\frac{dx}{dy})^2}dy$$

\section{Applications to Physics and Engineering}
\subsection{Hydrostatic Force}
Integral of the area of each slice times the pressure on that slice at that depth (pressure = density * depth)

\subsection{Center of Mass}
For a thin plate on the plane, the centroid is at $(\bar{x}, \bar{y})$ where:
$$ \bar{x} = \frac{1}{A} \int_a^b x(f(x)-g(x))dx $$
$$ \bar{y} = \frac{1}{2A} \int_a^b (f(x)^2 -g(x)^2) dx$$

\subsection{Rotating cross sections around line}
If cross section is completely outside of the line, volume = area * distanc traveled by cross section.

\chapter{Parametric Equations and Polar Coordinates}
\section{Parametric Equations}
\subsection{Tangents}
$$\frac{dy}{dx} = \frac{ \frac{dy}{dt} }{ \frac{dx}{dt} }$$
$$\frac{d^2y}{dx^2} = \frac{ \frac{d}{dt}(\frac{dy}{dx}) }{ \frac{dx}{dt} }$$

\subsection{Integrals}
\subsubsection{Normal Integral: Area from $\alpha \to \beta$}
$$\int_{\alpha}^{\beta}y(t)*\frac{dx}{dt}dt$$
\subsubsection{Arc Length}
$$L = \int_{\alpha}^{\beta} \sqrt{ \frac{dx}{dt}^2 + \frac{dy}{dt}^2 }dt$$
\subsubsection{Surface Area of Revolution: }
$$S = \int_{\alpha}^{\beta} 2\pi y \sqrt{\frac{dx}{dt}^2 + \frac{dy}{dt}^2}dt$$



\section{Polar Coordinates}
\subsection{Quick Info: }
\begin{enumerate}
\item Points are in form $(r, \theta)$
\item Origin denoted by $O$ or as "pole"
\item For point $(x, y)$ in cartesian space and point $(r, \theta)$:
\begin{itemize}
\item $x = r cos \theta$
\item $y = r sin \theta$
\item $tan \theta = \frac{x}{y}$
\end{itemize}
\end{enumerate}

\subsection{Symmetry}
If an equation is unchanged when $\theta \to -\theta$, it is symmetric about line $\theta = 0$
If an equation is unchanged when $r \to -r$ OR $\theta \to \theta + \pi$, it is symmetric about the pole.
If an equation is unchanged when $\theta \to \pi - \theta$ then the curve is symmetric about the line $\theta = \pi/2$ (a vertical line in polar coordinates)

\subsection{Tangents to Polar Curves}
Treat as parametric equation. Steps:
$$ x = r cos \theta = f(\theta) cos (\theta)$$
$$ y = r sin \theta = f(\theta) sin (\theta)$$

Then $\frac{dy}{dx}$ is just $\frac{dy}{d\theta}/\frac{dx}{d\theta}$

\section{Areas and Lengths in Polar Coordinates}
\subsection{Area}
\paragraph{Area of a Sector of a Circle}
$$A = \frac{1}{2}r^2\theta$$
\paragraph{Area Inside Curve: }
$$A = \int_a^b \frac{1}{2} (f(\theta))^2 d\theta $$
$$A = \int_a^b \frac{1}{2} r^2 d\theta $$

\subsection{Arc Length}
\subsubsection{Review of Parametric Arc Length}
$$L = \int_a^b \sqrt{r^2 + (\frac{dr}{d\theta})^2 } d\theta$$


\chapter{Infinite Sequences and Series}
\section{Sequences}
\paragraph{Definition: } List of numbers written in definite order. Notation is the same as set notation.
\subsection{Limits of Sequences}
\paragraph{A sequence has a limit $L$ if: } you can make $a_n$ as close as you want to $L$ by increasing $n$

You can encorporate ($\delta$) $\epsilon$ notation if you want to.

Also, if you can find a function that matches the sequence at all integer points, then you can just find the limit of the function using regular limit rules to find the limit of the sequence.

You can also disperse the a limit inside of a function.
$$\lim_{n \to \infty} sin(\pi/n) = sin( \lim_{n \to \infty} \pi /n )$$

\subsection{Definitions}
\paragraph{Monotonic sequences} are sequences that are either strictly increasing or strictly decreasing.
\paragraph{Bounded above} if no value of $n$ will make $a_n$ greater than $M$. Ditto bounded below.
\paragraph{Monotonic Sequence Theorem: } Every bounded, monotonic sequence is convergent.

\section{Series}
\paragraph{A series is: } a sum of a sequence (often infinite)
\subsection{Partial Sums}
$$s_1 = a_1, s_2 = a_1 + a_2, s_n = a_1 + ... + a_n$$
$$s_n = \sum_{i=1}^{n} a_i$$
\subsection{Infinite Series}
$$\sum_{n=1}{\infty} a_n = \lim_{a \to \infty} \sum_{i=1}^{n} a_i$$

\subsection{Geometric Series}
If $a_n = (a_1)r^n$, then $s_n = a + ar + ar^2 + ... + ar^{n-1}$

$$s_n = \frac{a(1-r^n)}{1-r}$$

\subsection{Test for Divergence}
If $\lim_{n \to \infty} a_n$ does not exist or equals $\infty$, then the infinite series of $a_n$ is divergent.

\subsection{Working with Series}
You can add and subtract series normally.
$$\sum (a_n + b_n) = \sum a_n + \sum b_n$$
$$\sum (c * a_n) = c \sum a_n$$

\subsection{Comparison Test}
Let $a_n$ be one series you DON'T know the limit of and $b_n$ be another. Make sure $a_n/b_n$ divides nicely. let $c = \lim_{n \to \infty}a_n/b_n$

Use $c$ as a comparator - if it tells you that $a$ is bigger than $b$ and $b$ is known to be divergent, then $a$ must be divergent, etc.

\subsection{P-Series}
A p-series is of the form: $$\sum_{n=1}^{\infty} 1/{n^p}$$
If $p > 1$, it converges. Otherwise it diverges.

\section{The Integral Test and Estimates of Sums}
\subsection{Integral Test}
It's not easy to find the sum of series except for geometric series and $\sum 1/[n(1+n)]$. 

To prove that a sum diverges, take the integral of the continuous function $f(n) = a_n$ from $1 \to \infty$. If the integral is divergent, then so is the sum. If the integral is convergent, so is the sum. Think about the geometric argument and draw out the boxes on the page if necessary (reimann sums). 

\subsection{Estimating The Sum of a Series}
Any partial sum of a series is an approximation of the infinite sum. We can get a good picture of how good the approximation is via the \texttt{remainder}.
$$R_n = s - s_n = a_{n+1} + a_{n+2} + ...$$
By the integral test (and intuition), the remainder is less than or equal to theintegral from $n$ to $\infty$
$$R_n = a_{n+1} + a_{n+2} + ... \leq \int_n^{\infty}f(x)dx$$
$$R_n \geq \int_{n+1}^{\infty}f(x)dx$$

In other words, the \paragraph{Remainder Estimate for the Integral Test} is:
$$ \int_{n+1}^{\infty}f(x) dx \leq R_n \leq \int_n^{\infty}f(x) dx$$

\section{Comparison Tests}
\subsection{The Limit Comparison Test}
Let $\sum a_n$ and $\sum b_n$ are series with positive terms. If $$lim_{n \to \infty} \frac{a_n}{b_n} = c$$ and c is a finite number $>$ 0, then both either converge or diverge.

\section{Alternating Series}
Convergence tests so far only apply to positive series. 
\subsection{Alternating Series Test}
$$\sum_{n=1}^{\infty} (-1)^{n-1} b_n = b_1 - b_2 + b_3 - b_4 + ... b_n > 0$$
\paragraph{If } the series obeys:
\begin{enumerate}
\item $b_{n+1} \leq b_n$ for all n
\item $lim_{n \to \infty} b_n = 0$
\end{enumerate}
Then the series is CONVERGENT.

\paragraph{Fun fact: } $s_{2n} \leq b_1$ for all n. 

\subsection{Estimating Sums for Alternating Series}
\paragraph{Alternating Series Estimation Theorem: } If $s = \sum (-1)^{n-1} b_n$ where $b_n$ > 0, and the series $b$ converges to zero and is strictly decreases, then:
$$ |R_n| = |s-s_n| \leq b_{n+1} $$

\section{Absolute Convergence and Ratio and Roots Test}
\subsection{Absolute Convergence}
\paragraph{Definition: } $\sum a_n$ is convergent is $\sum |a_n|$ converges.

\paragraph{Conditional Convergence: } When a series is convergent, but not absolutely so (depends on the operators between values in series).

\subsection{Ratio Test}
\begin{enumerate}
\item if $lim_{n\to\infty} |(a_{n+1}/a_n| = L < 1$, then the sequence is \textbf{absolutely convergent}
\item if the result of the above is $> 1$, then the sequence is \textbf{divergent}
\item if the result of the above is $= 1$, then the ratio test was \textbf{inconclusive}.
\end{enumerate}

\subsection{Root Test}
This is just the same as the ratio test, but as it turnt out, you can apply the ratio test to stuff inside nth roots and the same results will be forthcoming. 

\subsection{Rearrangements}
\begin{enumerate}
\item Any rearrangement of a \textbf{absolutely convergent} series is the same.
\end{enumerate}

Non-absolutely convergent series don't have the same sums necessarily when the order is changed. 

\section{Strategy for Series: }
\begin{enumerate}
\item Is it a p-series (form: $\sum 1/{n^p}$)? If p > 1, it converges, otherwise, it doesn't. 
\item Is it a geometric series (form: $\sum ar^{n-1}$? If $|r| < 1$, then it converges. Otherwise it diverges.
\item If it is similar to a p-series or geometric series, use a comparison test.
\item If you can tell that $\lim_{n\to\infty} a_n \neq 0$, use \textbf{Test for Divegence} 
\item If it's $\sum(-1)^{n-1}b_n$ or of a similar form, then alternating series tests are the way to go. 
\item If it involves factorials or other products (e.g. constants raised to some power), use the ratio test. 
\item Use the root test for if $a_n$ is of form $(b_n)^n$
\item If the corresponding integral is easy to evaluate, then use the integral test. 
\end{enumerate}

\section{Power Series}
Of the form: 
$$\sum_{n=0}^{\infty} c_n x^n = c_0 + c_1x + c_2x^2 + ... + c_nx^n$$

$x$ is a variable, $c_n$ is a sequence of coefficients.

Converges when $-1 < x < 1$ if all $c_n$ are one. 

\paragraph{Power Series "Centered at $a$" or "In $(x-a)$"} when:
$$\sum_{n=0}^{\infty} c_n(x-a)^n = c_0 + c_1(x-a) + c_2(x-a)^2 + ...$$
Always converges for $x = a$ because all terms after $c_0$ reduce to 0.

\subsection{3 Possibilities for Power Series}
\begin{enumerate}
\item The series converges only when x = a
\item The series converges for all x
\item \exists R so that if $|(x-a)| < R$ then the series will converge. R is "radius of convergence".
\end{enumerate}

\section{Representing Functions as Power Series}
$$\frac{1}{1-x} = 1 + x + x^2 + ... = \sum_{n=0}^{\infty}x^n$$ where |x|<1






\end{document}
