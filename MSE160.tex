\documentclass[a4paper,12pt]{report}
\begin{document}
\title{MSE160 Lecture Notes}
\author{Aman Bhargava}
\date{January 2018}
\maketitle

\chapter{Course Information}

Midterm is now Thursday February 28

\chapter{Molecular Chemistry}
\section{Lecture 2: Light and Electron Energies}

\subsection{Miscellaneous Facts}
\begin{itemize}
\item Most metals are ductile (easy to make wires out of) and malleable (easy to flatten out)
\item However, lead is ONLY malleable
\item The reason for which all materials have different properties is because of differences in atomic structures and how it affects bonding with other atoms.
\end{itemize}

\subsection{Characteristics of Light}
\begin{itemize}
\item Wavelength is the distance between one crest to another (or trough to another), $\lambda$
\item Frequency is the number of cycles per second a wave goes through, $v$
\item $\lambda * v = c$ where $c$ is the speed of light ($3.0*10^8  m/s$)
\item Amplitude is the intensity of the wave.
\item Visible light is just a small part of the EM spectrum.
\begin{itemize}
\item Long radio waves
\item AM Radio
\item FM and TV Radio
\item Radar
\item Microwaves
\item Infrared
\item Visible Light
\item UV
\item X-Rays
\item Gamma rays
\end{itemize}
\end{itemize}

\subsection{Double Slit Experiment}
This experiment showed that light can interfere with itself in a wave-like fashion. If a trough and a crest overlap, then they will destructively interfere. Two crests or two troughs will constructively interfere.
The experiment showed this because, when light was shined through two slits, an interference pattern would be shown on a screen on the other side of the slits.

\subsection{Photoelectric Effect}
This experiment showed that light also has particle-like behavior. $E_{photon} = hv_{photon}$ where $h = 6.63*10^-34$. Surplus energy from photon goes directly into kinetic energy of the electron.
\begin{itemize}
\item When light is shines on metals, electrons are ejected.
\item By the past model, higher amplitude should result in greater kinetic energy of ejected electrons.
\item However, it seemed to depend on frequency instead.
\item Even with low amplitude light, if it was of high enough frequency, the kinetic energy of the ejected electrons was the same. The only difference was the rate of ejections.
\item Conclusion: Light comes in packets called "photons", and the energy of each photon is given by the equation above (linearly dependent on frequency). 
\end{itemize}
Conclusion: light acts like a wave and a particle (wow!).

\subsection{Further Information on Wave-Particle Duality}
Generally speaking, light behaves like a wave when the scale of interraction is large (i.e. light going through a prism - the prism is orders of magnitude larger than the wavelength of light). Meanwhile, light behaves more like a particle when interracting with things like atoms (atoms are roughly on the same order of magnitude as the wavelength of most light).

\subsection{Electron Energies}
\begin{itemize}
\item Electrons have discrete energy states where higher energy states have higher potential energy and lower states have lower potential energies.
\item Ground state is the lowest energy state that the electron can have in an atom, while excited state(s) are any states that are higher.
\item Convention: free electrons not bound to atoms have 0 energy (like potential gravitational energy).
\begin{itemize}
\item Electrons bound to atoms have $E < 0$, and ground state has $E << 0$.
\end{itemize} 
\item Photons can be absorbed to raise an electron to a higher energy state.
\item Photons are emitted when electrons fall down to a lower energy state.
\end{itemize}
\subsubsection{Atomic Spectra}
When atoms are in a place where their electrons are constantly being excited to above the ground state and then are falling back down, they emit their own signature spectrum due to the differences in energies between that particular atom's energy levels. You can solve for the frequency of the light emitted by electrons falling from one energy level to the other with $E = hv$.

\subsection{Models for Atoms}
\begin{itemize}
\item Rutherford model had electrons randomly hovering/orbiting a nucleus.
\item Due to the observation of distinct atomic spectra, Bohr predicted that there must be distinct energy levels that each atoms has.
\item Massive step forward because it was the first time that we could predict the behavior of atoms.
\item It had a lot of issues though!
\begin{itemize}
\item It did not generalize well to non-hydrogen atoms.
\item Didn't add up with the fact that magnetic/static fields would mess up spectral lines
\item As well, it conflicted with Maxwell's theory (classical electromagnetic theory)
\begin{itemize}
\item A moving charge should emit energy/light
\item Therefore, by classical electromagnetic theory, electrosn should lose energy until they crash into the nucleus.
\item Bohr did not realize that electrons don't really behave like particles.
\end{itemize}
\end{itemize}
\end{itemize}

\subsection{Double-Slit Experiment with Electrons}
When you shoot electrons through double slits, they ALSO show an interference pattern! This indicates that electrons have wave-like properties.
\begin{itemize}
\item De Broglie arrgued that all objects have a wavelength associated with them.
\item Given by $\lambda_particle = \frac{h}{p} = \frac{h}/{mv}$
\item The reason for which we don't see wave properties in large objects is because mass is in the denominator, so wavelength is unimaginably small for things larger than subatomic particles.
\end{itemize}

\subsection{Equations}
\begin{eqnarray}
E = hv \\
E_{kinetic} = \frac{1}{2}mv^2 \\
\lambda = \frac{hc}{E} \\
\lambda = \frac{h}{mv} \\
c = 3.8*10^8 m/s 
\end{eqnarray}

\subsection{Applications}
\subsubsection{Electron Microscopes}
\begin{itemize}
\item Since electrons have wave properties, we can use them to get images of very small things.
\item Light microscopes have limited resolving power because wavelength of light is kind of big compared to atoms.
\item So we can get really really nice images of really small objects with electron microscopes.
\end{itemize} 
\subsubsection{Quantum Technology}
\begin{itemize}
\item Quantum dots are semiconductor molecules that have very large absorption spectra but very narrow emission spectrum.
\item As well, we can tailor molecules to have practically any emission frequency
\item With this, we can image where molecules go in the body.
\item If we can attach ligands (side molecules) to the quantum dots, we can very specifically target things like cancer.
\end{itemize}
\chapter{Materials}


\end{document}
